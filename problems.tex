\documentclass{article}
\usepackage[cm]{fullpage}
\usepackage{multicol}
\setlength\parindent{0pt}
\begin{document}
\begin{multicols*}{2}
\section*{1A}

\textbf{1a}
A numerical variable is a variable representing a quantity, rather than a quantity. An example of a numerical variable is a height of 190cm.

\textbf{1b}
A categorical variable is a variable representing a quality, rather than a quantity. An example of a categorical variable is a number of 1, 2, or 3, to represent underweight, middleweight, or overweight respectively.

\textbf{2}
Discrete and continuous: Discrete variables represent quantities that are counted. Continuous variables represent quantities that are measured.

\textbf{3a}
Numerical variable (continuous)

\textbf{3b}
Numerical variable (discrete)

\textbf{3c}
Numerical variable (continuous)

\textbf{3d}
Categorical variable

\textbf{3e}
Categorical variable

\textbf{3f}
Numerical variable (discrete)

\textbf{3g}
Categorical

\textbf{3h}
Numerical variable (discrete)

\textbf{3i}
Numerical variable (continuous)

\textbf{3j}
Categorical

\textbf{3k}
Categorical

\textbf{3l}
Numerical variable (discrete)

\textbf{3m}
Categorical

\textbf{3n}
Numerical variable (discrete)

\textbf{4}
Height: numerical; Weight: numerical; Age: numerical; Sex: categorical; Plays sport: categorical; Pulse rate: numerical

\section*{1B}

\textbf{1a}


\end{multicols*}
\end{document}