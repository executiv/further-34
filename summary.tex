\documentclass[a4paper,11pt]{article}

\usepackage{tkz-euclide, textcomp, gensymb, amssymb, outlines, amsmath}
\usetkzobj{all}
\usetikzlibrary{arrows.meta}
\usepackage[makeroom]{cancel}

\title{\LARGE \scshape{2014 Curriculum Summary} \\ \Huge \scshape{Further Mathematics}}
\author{\Large \scshape{Tyler Fernando} \\ \small \scshape{Yarra Valley Grammar}}
\date{}

\begin{document}
\maketitle
\thispagestyle{empty}
\pagestyle{empty}

\tableofcontents

\newpage

\pagestyle{plain}

\setcounter{page}{1}

\section{Organising and displaying data}
\begin{outline}

\0
\subsection{Classifying data}

\0
\subsection{Organising and displaying categorical data}

\0
\subsection{Organising and displaying numerical data}

\0
\subsection{What to look for in a histogram}

\0
\subsection{Stem-and-leaf plots and dot plots}

\end{outline}

\newpage

\section{Summarising numerical data: the median, range, IQR and box plots}
\begin{outline}

\0
\subsection{Will less than the whole picture do?}

\0
\subsection{The median, range and interquartile range (IQR)}

\0
\subsection{The five-number summary and the box plot}

\0
\subsection{Relating a box plot to distribution shape}

\0
\subsection{Interpreting box plots: describing and comparing distributions}

\end{outline}

\newpage

\section{Summarising numerical data: the mean and standard deviation}
\begin{outline}

\0
\subsection{The mean}

\0
\subsection{Measuring the spread around the mean: the standard deviation}

\0
\subsection{The normal distribution and the 68-95-99.7\% rule: giving meaning to the standard deviation}

\0
\subsection{Standard scores}

\0
\subsection{Populations and samples}

\end{outline}

\newpage

\section{Displaying and describing relationships between two variables}
\begin{outline}

\0
\subsection{Investigating the relationship between two categorical variables}

\0
\subsection{Using a segmented bar chart to identify relationships in tabulated data}

\0
\subsection{Investigating the relationship between a numerical and a categorical vehicle}

\0
\subsection{Investigating the relationship between two numerical variables}

\0
\subsection{How to interpret a scatterplot}

\0
\subsection{Calculating Pearson's correlation coefficient, r}

\0
\subsection{The coefficient of determination}

\0
\subsection{Correlation and causality}

\0
\subsection{Which graph?}

\end{outline}

\newpage

\section{Regression: fitting lines to data}
\begin{outline}

\0
\subsection{Least squares regression line: the theory}

\0
\subsection{Calculating the equation of the least squares regression line}

\0
\subsection{Performing a regression analysis}

\0
\subsection{A graphical approach to regression: the three median line}

\0
\subsection{Extrapolation and interpolation}

\end{outline}

\newpage

\section{Data transformation}
\begin{outline}

\0
\subsection{Data transformation}

\0
\subsection{Transforming the x-axis}

\0
\subsection{Transforming the y-axis}

\0
\subsection{Choosing and applying the appropriate transformation}

\end{outline}

\newpage

\section{Time series}
\begin{outline}

\0
\subsection{Time series data}

\0
\subsection{Smoothing a time series plot (moving means)}

\0
\subsection{Smoothing a time series plot (moving medians)}

\0
\subsection{Seasonal indices}

\0
\subsection{Fitting a trend line and forecasting}

\end{outline}

\newpage

\section{Revision of the Core}
\begin{outline}

\0
\subsection{Displaying, summarising and describing univariate data}

\0
\subsection{Displaying, summarising and describing relationships in bivariate data}

\0
\subsection{Regression and data transformation}

\0
\subsection{Time series}

\0
\subsection{Extended-response questions}

\end{outline}

\newpage

\section{Arithmetic and geometric sequences}
\begin{outline}

\0
\subsection{Sequences}

\0
\subsection{Arithmetic sequences}

\0
\subsection{The nth term of an arithmetic sequence and its applications}

\0
\subsection{The sum of an arithmetic sequence and its applications}

\0
\subsection{Geometric sequences}

\0
\subsection{The nth term of a geometric sequence}

\0
\subsection{Applications modelled by geometric sequences}

\0
\subsection{The sum of the terms in a geometric sequence}

\0
\subsection{The sum of an infinite geometric sequence}

\0
\subsection{Rates of growth of arithmetic and geometric sequences}

\end{outline}

\newpage

\section{Difference equations}
\begin{outline}

\0
\subsection{Introduction}

\0
\subsection{The relationship between arithmetic and geometric sequences and difference equations}

\0
\subsection{First-order difference equations}

\0
\subsection{Solving first-order difference equations that generate arithmetic sequences}

\0
\subsection{Solving difference equations that generate geometric sequences}

\0
\subsection{Solution of general first-order difference equations}

\0
\subsection{Summary of first-order difference equations}

\0
\subsection{Applications of first-order difference equations}

\0
\subsection{The Fibonacci sequence}

\end{outline}

\newpage

\section{Revision: Number patterns and applications}
\begin{outline}

\0
\subsection{Multiple-choice questions}

\0
\subsection{Extended-response questions}

\end{outline}

\newpage

\section{Geometry}
\begin{outline}

\0
\subsection{Properties of parallel lines: a review}

\0
\subsection{Properties of triangles: a review}

\0
\subsection{Properties of regular polygons: a review}

\0
\subsection{Pythagoras' theorem}

\0
\subsection{Similar figures}

\0
\subsection{Volumes and surface areas}

\0
\subsection{Areas, volumes and similarity}

\end{outline}

\newpage

\section{Trigonometry}
\begin{outline}

\0
\subsection{Defining sine, cosine and tangent}

\0
\subsection{The sine rule}

\0
\subsection{The cosine rule}

\0
\subsection{Area of a triangle}

\end{outline}

\newpage

\section{Applications of geometry and trigonometry}
\begin{outline}

\0
\subsection{Angles of elevation and depression, bearing, and triangulation}

\0
\subsection{Problems in three dimensions}

\0
\subsection{Contour maps}

\end{outline}

\newpage

\section{Revision: Geometry and trigonometry}
\begin{outline}

\0
\subsection{Multiple-choice questions}

\0
\subsection{Extended-response questions}

\end{outline}

\newpage

\section{Constructing and interpreting linear graphs}
\begin{outline}

\0
\subsection{The gradient of a straight line}

\0
\subsection{The general equation of a straight line}

\0
\subsection{Finding the equation of a straight line}

\0
\subsection{Equation of a straight line in intercept form}

\0
\subsection{Linear models}

\0
\subsection{Simultaneous equations}

\0
\subsection{Problems involving simultaneous linear equations}

\0
\subsection{Break-even analysis}

\end{outline}

\newpage

\section{Graphs}
\begin{outline}

\0
\subsection{Line segment graphs}

\0
\subsection{Step graphs}

\0
\subsection{Non-linear graphs}

\0
\subsection{Relations of the form $y = kx^n$ for $n = 1, 2, 3, -1, -2$}

\0
\subsection{Linear representation of non-linear relations}

\end{outline}

\newpage

\section{Linear programming}
\begin{outline}

\0
\subsection{Region defined by an inequality}

\0
\subsection{Regions defined by two inequalities}

\0
\subsection{Feasible regions}

\0
\subsection{Objective functions}

\end{outline}

\newpage

\section{Revision: Graphs and relations}
\begin{outline}

\0
\subsection{Multiple-choice questions}

\0
\subsection{Extended-response questions}

\end{outline}

\newpage

\section{Principles of financial mathematics}
\begin{outline}

\0
\subsection{Percentage change}

\0
\subsection{Simple interest}

\0
\subsection{Compound interest}

\0
\subsection{Reducing balance loans}

\end{outline}

\newpage

\section{Applications of financial mathematics}
\begin{outline}

\0
\subsection{Percentage changes and charges}

\0
\subsection{Bank account balances}

\0
\subsection{Time payments (Hire purchase)}

\0
\subsection{Inflation}

\0
\subsection{Depreciation}

\0
\subsection{Applications of finance solvers}

\end{outline}

\newpage

\section{Revision: Business-related mathematics}
\begin{outline}

\0
\subsection{Multiple-choice questions}

\0
\subsection{Extended-response questions}

\end{outline}

\newpage

\section{Undirected graphs}
\begin{outline}

\0
\subsection{Introduction and definitions}

\0
\subsection{Planar graphs and Euler's formula}

\0
\subsection{Complete graphs}

\0
\subsection{Euler and Hamilton paths}

\0
\subsection{Weighted graphs}

\end{outline}

\newpage

\section{Directed graphs}
\begin{outline}

\0
\subsection{Introduction, reachability and dominance}

\0
\subsection{Network flows}

\0
\subsection{The critical path problem}

\0
\subsection{Allocation problems}

\end{outline}

\newpage

\section{Revision: Networks and decision mathematics}
\begin{outline}

\0
\subsection{Multiple-choice questions}

\0
\subsection{Extended-response questions}

\end{outline}

\newpage

\section{Matrices and applications I}
\begin{outline}

\0
\subsection{What is a matrix?}

\0
\subsection{Using matrices to represent information}

\0
\subsection{Matrix arithmetic: addition, subtraction and scalar multiplication}

\0
\subsection{Matrix arithmetic: the product of two matrices}

\end{outline}

\newpage

\section{Matrices and applications II}
\begin{outline}

\0
\subsection{The inverse matrix}

\0
\subsection{Applications of the inverse matrix: solving simultaneous linear equations}

\0
\subsection{Matrix powers}

\0
\subsection{Transition matrices and their applications}

\end{outline}

\newpage

\section{Revision: Matrices and applications}
\begin{outline}

\0
\subsection{Multiple-choice questions}

\0
\subsection{Extended-response questions}

\end{outline}

\end{document}